\section{Background}

\subsection{OpenGL and the GPU}
Games (and most any real-time applications) leverage the impressive power of modern graphics processing units (GPUs).
In the past, GPUs were almost entirely used for graphics processing as the name would suggest, however general purpose GPU computing (GPGPU) has also become popular (\todo{cite}).
OpenGL is an application programming interface (API) that allows developers to communicate with graphics hardware.

As graphics hardware has progressed, OpenGL has also needed to expand, and thus has different versions.
As of the time of this writing, OpenGL is up to version 4.4 (\todo{cite}).
However, one must keep in mind that OpenGL versions only specify a minimum specification that graphics hardware must meet.
For example, an OpenGL 2 capable graphics card may still be able to use some OpenGL 3 features.
Therefore, when using OpenGL, it is common to target features instead of OpenGL versions (e.g., check for shader support instead of checking for OpenGL 2 support).

In earlier versions of OpenGL, a fixed graphics pipeline was defined.
This pipeline could be setup, used and influenced by commands, but certain parts (e.g., the algorithm to determine pixel color) were still fixed.
Eventually, parts of the pipeline became programmable.
The programs written to control these parts of the pipeline are called shaders.
OpenGL 4.4 specifies four shader types: vertex, geometry, fragment, and compute (\todo{cite?}).
Primarily GLSL is used to write these shaders, but other languages such as Cg are supported (\todo{cite?}).
Of interest to this work are the vertex and (partially) the fragment shader.
A vertex shader controls vertex transformations (e.g., the model to view space coordinate conversion), and a fragment shader controls how a fragment (think pixel) is colored or shaded.

\subsubsection{GPU/CPU Interaction}
When using a GPU, the CPU is still used.
This means that two processors are being used asynchronously and in parallel.
The CPU is used to issue commands to the GPU (usually in some form of render step)
Usually the CPU can continue to work while the GPU processes the commands it was given.
However, the GPU driver can also block the CPU if, for example, the GPU has too much work to do.
This gives two general categories of bottlenecks in a graphics application:
\begin{enumerate}
 \item The CPU is not issuing commands to the GPU fast enough. This results in the GPU being underutilized, and is known as being CPU-bound.
 \item The CPU is too far ahead of the GPU and the GPU drivers cause the CPU to wait. In this case the CPU is being underutilized, and this is known as being GPU-bound.
\end{enumerate}
Improving the performance of code on running on the CPU will not improve performance if the CPU is just going to wait longer for the GPU to finish (GPU-bound).
And similarly, if the application is CPU-bound, giving the GPU less work will also not improve performance.
Therefore, it is important to identify the bottleneck and keep the load on the CPU and GPU balanced for the best performance.

\subsection{Skeletal Mesh Animation}
\ifsummaries
\begin{itemize}
 \item Discuss bones/skeleton/armature and meshes
 \item Bone rotation and position stored in a 4x4 transform matrix
\end{itemize}
\fi

One popular method used to drive character animations is to use a set of bones (also known as joints) called a skeleton (or armature) to deform a mesh.
An example of this can be seen in figure~\ref{fig:ex_skmesh}.
\fig{imgs/sinbad_skeletal_mesh}{0.5}{An example skeletal mesh}{ex_skmesh}
Once a pose (a set of transforms for each bone) has been determined, an operation called skinning is used to deform the mesh to the pose.
Transformations are usually represented as a four-by-four matrix, which encodes rotation, translation and scale.
The pose transformations are relative to an initial ``rest'' pose.
Each vertex in the mesh has a set of weights, which sum to one, to determine the amount of influence each bone has on that vertex.
% To deform the mesh using the skeleton, iterate over every vertex and transform the vertex using a weighted sum of the four-by-four transform (rotation, translation and scale) matrices of the bones that influence the given vertex.
% In other words, a vertex's transform can be defined as:
To determine each vertex's new skinned position and normal, equations~\ref{eq:skinning_p}~and~\ref{eq:skinning_n} can be used.

\begin{equation}
 \label{eq:skinning_p}
  P^\prime = \sum_b{\left(w_bT_bP\right)}
\end{equation}

\begin{equation}
 \label{eq:skinning_n}
 N^\prime = \sum_b{\left(w_bT_{rb}N\right)}
%  T_v = \sum_b{I_bT_b}
\end{equation}

For the position equation (equation~\ref{eq:skinning_p}) $P$ is the initial, rest position of the vertex, $b$ is a bone, $w_b$ is the scalar influence (i.e., the weight) of $b$ on the vertex and $T_b$ is the four-by-four transform matrix of the bone.
The normal equation (equation~\ref{eq:skinning_n}) is similar except we use the normal's resting rotation ($N$) and the three-by-three rotation part of the bone transformation matrix as $T_{rb}$.
These calculations must be done every time the pose of the skeleton is updated to create a new set of positions and normals for a mesh's vertexes.
When rendering the mesh this also requires the vertex data to be re-sent to the graphics card if the skinning calculations were done on the CPU (software skinning).

Figure~\ref{fig:sk_anim_ex} shows an example skeletal mesh animation in Blender, which uses a right-handed, Z-up coordinate system.
This means that the up axis is the positive Z axis, the right is the positive X axis, and going into the screen is the positive Y axis.
The example is a rectangular prism centered on the origin.
\fig{imgs/sk_anim_ex}{0.5}{An example skeletal mesh animation}{sk_anim_ex}
Using equation~\ref{eq:skinning_p}, a skinned position can be determined for the vertex labeled $V$ due to the animation shown:
\begin{gather}
 \label{eq:skinning_ex}
 V = \left(\begin{array}{c}0.0\\-1.0\\1.0\\1.0\end{array}\right)
 T_{b1} = \left(\begin{array}{cccc}1.0&0.0&0.0&0.0\\0.0&1.0&0.0&0.0\\0.0&0.0&1.0&0.0\\0.0&0.0&0.0&1.0\end{array}\right)
 T_{b2} = \left(\begin{array}{cccc}0.7071&0.0&-0.7071&0.0\\0.0&1.0&0.0&0.0\\0.7071&0.0&0.7071&0.0\\0.0&0.0&0.0&1.0\end{array}\right)\nonumber\\
 w_{b1} = 0.5\nonumber\\
 w_{b2} = 0.5\nonumber\\\nonumber\\
 V^\prime = \sum_b{\left(w_bT_bV\right)}\nonumber\\
 V^\prime = w_{b1}T_{b1}V + w_{b2}T_{b2}V\nonumber\\
 V^\prime = 0.5\left(\begin{array}{cccc}1.0&0.0&0.0&0.0\\0.0&1.0&0.0&0.0\\0.0&0.0&1.0&0.0\\0.0&0.0&0.0&1.0\end{array}\right)\left(\begin{array}{c}0.0\\-1.0\\1.0\\1.0\end{array}\right) + 0.5\left(\begin{array}{cccc}0.7071&0.0&-0.7071&0.0\\0.0&1.0&0.0&0.0\\0.7071&0.0&0.7071&0.0\\0.0&0.0&0.0&1.0\end{array}\right)\left(\begin{array}{c}0.0\\-1.0\\1.0\\1.0\end{array}\right)\nonumber\\
 V^\prime = \left(\begin{array}{c}-0.35355\\-1.0\\0.85255\\1.0\end{array}\right)\begin{array}{c}(left)\\(forward)\\(up)\\ \\\end{array}
\end{gather}


A technique referred to as hardware skinning moves the skinning step from the CPU to the GPU.
This also allows the skinning calculation to benefit from the impressive parallelization offered by modern graphics hardware.
Another benefit is that the skeletal mesh can be treated as static geometry since all of the deformations are done on the GPU.
This allows options such as display lists and vertex buffer objects\cite{ARB_texture_buffer_object} to be used to increase rendering performance.

The differences in program flow between software and hardware skinning are shown in figure~\ref{fig:skinned_update}.

\fig{uml/skinned_update}{0.5}{UML Activity diagram for skinned animations}{skinned_update}

As can be seen in figure~\ref{fig:skinned_update}, instead of sending new vertex data every frame, the pose data must be sent every frame.
This means, as far as memory bandwidth, $number\_of\_verts*sizeof(vertex)$ is being traded for $number\_of\_bones*sizeof(bone_transform)$.
The size of a vertex in the BGE can range from between six floats (position and normal) to thirty-one floats (position, normal, color, and various texture coordinates).
Using just position, normals and UV texture coordinates results in only eight floats and is feasible for a simple character.
Bone transformations, as mentioned earlier, are a four-by-four matrix, which is sixteen floats.
As for the number of vertexes and bones, documentation for Unity3D (a game engine) notes that skeletons will usually have between sixteen and sixty bones, and meshes will usually have between 1,500 to 4,000 vertexes, but possibly upwards of 5,000 to 7,000 vertexes\cite{unity_opt_char}.
Using thirty bones as suggested in \cite{unity_opt_char} and an average of 2,750 vertexes, the total memory, $M$, needed for vertex data is (assuming four bytes per float):
\begin{align*}
 M &= number\_of\_verts*sizeof(vertex)\\
 M &= 2,750 * 8 * sizeof(float)\\
 M &= 88,000\ bytes
\end{align*}

and for pose data is:

\begin{align*}
 M &= number\_of\_bones*sizeof(bone\_transform)\\
 M &= 30 * 16 * sizeof(float)\\
 M &= 1,920\ bytes
\end{align*}
In other words, hardware skinning will typically not use more memory bandwidth than software skinning.

Hardware skinning is usually implemented as a vertex shader.
An example GLSL vertex shader from version 9.52 of NVIDIA's graphics SDK\cite{nvidiasdk} is shown in listing~\ref{lst:nv_skinning}.

\includeglsl{nv_skinning}{Example vertex shader for hardware skinning from NVIDIA's graphics SDK}

Attributes are per-vertex values while uniforms are global values.
Typically uniforms will be changed per-material or per-mesh.
The NVIDIA example has a maximum of four bones influencing each vertex, which allows weights and indexes (into the boneMatrices array) to be stored as vec4 (four floats) data types.
It also imposes a limit of thirty bones for the whole skeleton.
%Currently the BGE imposes no such limits, therefore other methods of transferring the data from the CPU to GPU that do not impose a preset limit may have to be explored.
In the NVIDIA shader, the mat44 variable stores a bone's four-by-four transform matrix, which is multiplied against the current vertex's vec4 position value (the part inside the summation of equation~\ref{eq:skinning_p}).
The mat33 variable is a three-by-three orientation matrix constructed from the bone's transform matrix, and it is multiplied against the current vertex's vec3 normal value (the part inside the summation of equation~\ref{eq:skinning_n}).
A for loop is used for the summation, and the resulting vectors are added back to the initial position and normal vectors.

\subsection{Blender and the BGE}
\ifsummaries
\begin{itemize}
 \item Overview of the BGE animation and skinning code
 \item BGE prefixes
 \item Blender's DNA and RNA systems
\end{itemize}
\fi

\subsubsection{Class Prefixes}
Blender and the BGE are broken down into many modules.
Functions and classes from these modules often have a prefix to help identify them (these prefixes also act as a poor-man's namespace to avoid name conflicts).
In the BGE the following prefixes can be found:

\begin{description}
 \item[BL] Items closely tied to Blender. Most of these items are part of the code that converts Blender data to BGE data.
 \item[SCA] Generic game engine code. This handles a lot of generic logic and most of the BGE's logic brick system (a visual programming system). The prefix comes from the BGE's Sensor, Controller, Actuator architecture for logic bricks.
 \item [KX] These items tend to be BGE specific code. The name comes from Ketsji, which was an old name for the BGE.
 \item [KX\_SCA] These items blur the line between SCA and KX items. They should probably belong in one prefix or the other.
 \item [SG] These items belong to the BGE's scenegraph code.
 \item [RAS] These items belong to the BGE's rasterization, or rendering, code. This code is usually referred to as the ``rasterizer.''
 \item [PHY] These items belong to the BGE's generic physics interface.
\end{description}

Some prefixes from Blender that one might see used in BGE code include:

\begin{description}
 \item [BLF] Blender font handling code.
 \item [BKE] Blender kernel, code. This contains most of the code to actually manipulate Blender data.
 \item [BLI] Contains useful data structures such as linked lists, trees, etc.
 \item [BLO] Blender file loading code.
 \item [GPU] Blender GPU rendering code that is primarily used for the viewport. This module is sometimes referred to as bf\_gpu when talking about rendering to differentiate it from general gpu code/programming (e.g., OpenGL).
 \item [DNA] Blender struct definition. Refer to section~\ref{sec:bf_dna_rna} for more details.
 \item [RNA] Description layer for DNA. Refer to section~\ref{sec:bf_dna_rna} for more details.
\end{description}

\subsubsection{Blender and DNA/RNA}
\label{sec:bf_dna_rna}

Blender contains two systems for storing and manipulating data, these are DNA and RNA. DNA is responsible for storing (and ultimately saving) Blender data, while RNA describes the DNA data and how it can be accessed.
In other words, DNA values (stored as C structs) are modified via RNA (functions).
Blender's animation system uses this RNA for writing out new pose data.
Unfortunately looking up RNA values and modifying them were never optimized for speed, and sometimes these operations can be slow.

\subsubsection{BGE Overview}
\label{sec:bge_general_overview}

\fig{uml/bge_activity}{0.5}{UML activity diagram for the BGE's main loop}{bge_main_loop}

Figure~\ref{fig:bge_main_loop} shows a simplified version of the BGE's main loop.
As can be seen, every frame the BGE has roughly four main tasks: update logic, update physics, update animations, and render.
When updating logic, the engine evaluates and executes any logic bricks or Python scripts that a user may have setup.
The physics engine (the BGE uses Bullet) is then given the chance to evaluate the physics scene and step its simulations.
Animations are updated next, which is further explained in section~\ref{sec:bge_anim_overview}.
Finally, the scene must be rendered.
This involves determining what needs to be rendered (objects that will not be rendered this frame are ``culled''), and issuing the appropriate commands to have the GPU render the scene.
Excluding the actual rasterization of the scene, everything is performed on the CPU (including culling and mesh deformations), and in a single thread.

\subsubsection{BGE Animation Code}
\label{sec:bge_anim_overview}

The code for updating animations can be seen in listing~\ref{lst:anim_code}.

\includecpp{anim_code}{BGE animation code}

To handle animations, the BGE makes use of various classes, which are shown in figure~\ref{fig:bge_class}.
For discussion purposes, the prefixes will be left off of the class names.

\fig{uml/bge_class}{0.5}{UML class diagram of classes needed for animations in the BGE}{bge_class}

The Scene class contains an aggregate, called m\_objectlist, of all of the GameObjects in the scene (CListValue is a wrapper around C++'s vector class).
Some of those objects can also be present in m\_animatedlist, which is iterated in Scene::UpdateAnimations().
ArmatureObjects represent skeletons that we may or may not want to animate, and DeformableObjects are ones with deform-able meshes.
A handful of Deformers exist in the BGE, but the SkinDeformer is the only one of interest for skeletal mesh animations.
The SkinDeformer will deform the mesh based off of the current pose of an ArmatureObject (usually the mesh's parent object).
A Deformer is updated during the render stage while ArmatureObjects are updated during the animation stage.

The ArmatureObject contains Blender data in the form of bArmature and bPose pointers.
Every ArmatureObject that uses the same skeleton will also point to the same bArmature.
All ArmatureObjects make copy of their bArmature's bPose.
In order to actually update the pose, the bArmature's bPose pointer is saved and replaced with the ArmatureObject's bPose, and then some Blender functions are called to animate the bArmature.
The bArmature's original pointer is then restored.
The code that performs this update (as well as some blending) is shown in listing~\ref{lst:anim_pose_update}.

\includecpp{anim_pose_update}{Pose update performed as part of BL\_Action::Update()}

The SkinDeformer contains two functions for handling the mesh deformations that are exposed to the user as ``Vertex Deformers'': BlenderDeformVerts() and BGEDeformVerts().
Originally the SkinDeformer would make calls into Blender code to handle the deformation.
Later a ``BGE Vertex Deformer'' (the original code was used to create the BlenderDeformVerts() function) was added to focus on speed over accuracy, and it decreases the frame time of an animation heavy scene by about 30\%.
However, it lacks some features such as support for Blender's B-Bones, and it has less accurate normal calculations.
These vertex deformers are called as part of SkinDeformer::Update(), which, in turn, is called as part of SkinDeformer::Apply().
This makes SkinDeformer::Apply() the function to call to kickoff skinning in the BGE.

When setting up a skeletal mesh in Blender, a mesh and skeleton are first both created.
Then, the mesh is ``parented'' to the skeleton.
Thus, a mesh has at most one skeleton deforming it, but a skeleton can deform many meshes.
However, in most cases this is simply a one-to-one relationship.

Other types of animations other than skeletal mesh animations can be performed.
These include changing an object's color, changing an object's transform (position, rotation and/or scale), and other minor animations.
Since these animations are simple (usually updating a single vector or matrix), they are usually much faster than animating a skeletal mesh.
As such, this work will only focus on improving the performance of skeletal mesh animations, and ignore the performance of other types of animations.

% \subsection{OpenMP}
% \ifsummaries
% \begin{itemize}
%  \item Used for threading
%  \item Uses pragmas and requires compiler support
% \end{itemize}
% \fi
%
% \todo{Fill out this section}

%% Currently covered by the Skeletal Mesh Animation subsection
% \subsection{Hardware Skinning}
% \begin{itemize}
%  \item Moves the skinning step to the GPU
%  \item Allows skeletal mesh to be treated as ``static'' geometry, which is faster to render
%  \item Usually implemented as a vertex shader
% \end{itemize}
%
